\documentclass{article}
\title{Battleship AI: documentation}
\author{Vardhan Gupta}
\date{may-june 2017}
\begin{document}
   \maketitle
   \tableofcontents
   \section{Introduction}
   This project aims to make an AI which can play the Battleship game intelligently. the code for this is written in python and we aim to use the pygame library of python to implement the graphical part of the game. We also try to look into various algorithms to play the game and finish it efficiently.
   
   \section{the battleship game}
   Battleship is a board game played between two players. each player is supposed to place ships in two different oceans i.e 10x10 grids. the ships are of different sizes namely, aircraft carrier (5x1), battleship (4x1), destroyer (3x1), cruiser (3x1) and a patrol boat (2x1). they may be placed anywhere on the grids either horizontally or vertically. they may never overlap with each other but can be placed adjacent to each other.The players take turns alternatively. On each turn a player is supposed to fire (guess a point on opponent's grid) at the opponents ships which may result in either a 'miss' or a 'hit'. when all the points of a ship have been hit, it results in sinking of that ship. the first player to sink all of opponent's ships is declared the winner.
   
   \section{work done so far}
   so far a we have made a basic console based version of the game. it is a very basic version which uses quite simple algorithms.
   
   the algorithm used here is called hunt/target with parity. in this algorithm, we first randomly shoot in a checker board like pattern till a hit is found. once a hit has been found we try to sink that ship on which the hit was found. if there are no hits lift then and everthing is either in 'sink' state or 'miss' state or 'unguessed' state then we go back to the random shooting.
   
   Once a hit is found, the direction with most unguessed points is chosen to shoot at.


	the functions made so far are:
	\begin{itemize}
   		
   		\item placeship, places ships on the board.
   		\item checksink, checks the sinking status of any ship after a hit.
   		\item shootingdirection, shoots in a given direction in order to sink a ship.
   		\item targetmode, finds the best possible direction to shoot in and recalls itself if a hit is left on the board after sink a ship
	
	\end{itemize}
	
	such an algorithm gave an average of around 56 moves to beat a random board when tested for 1000 games.
	
	the next algorithm used calculates the probability of every unguessed point in the grid having a ship. the point with the best probability is targeted. if multiple points have the maximum probability then a \textit{neighbour score} is calculated for each such point and the point with maximum neighbour score is shot at. if still multiple points qualify, then the point on the checker board like pattern is shot at. if still multiple points qualify, then any one of these points is chosen at random.
	
	Also, after a hit is found, instead of shooting in the direction with most unguessed points, we add the no. of unguessed points of 'up' with 'down' and 'left with 'right' and compare them. the directions which add upto the bigger one of them is chosen and the best direction is now the direction with the most unguessed points of these two.
	
	the functions made/changed for this algorithm are:
	\begin{itemize}
	
		\item calcprob, calculates the probability of having a ship of a particular size over the entire board.
		\item targetmode, changes were made as mentioned above for obtaining better shooting direction after a hit.
		\item nscore, calculates the neighbour score of a given point.
	\end{itemize}
	
	this approach was a significant improvement over the previous one as it resulted in an average of about 44 moves every 1000 games.
	
	I have also made a file \textit{gamemaster.py} which imports the code of the original file, \textit{battleship.py} and uses it to actually play the game.
   
   \section{future plans}
   we aim to use the pygame library on this to make the game graphical instead of console based.
  
  Also, we plan to make the placement of ships intelligent, by keeping in the order of guesses made by the opponent in the previous games.
  
  We will keep trying to refine the algorithm and make it more efficient.
\end{document}
